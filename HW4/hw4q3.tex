\documentclass{article}
\usepackage{graphicx,amsmath,amssymb} % Required for inserting images

\begin{document}

\section*{3. Find the natural cubic spline $S(x)$ that interpolates:}
\[(1,3),(2,1),(3,2),(4,3),(5,2).\]

\subsection*{Cubic Spline Form}  
Each spline segment \( S_j(x) \) is a cubic polynomial of the form:
\[
S_j(x) = a_j + b_j (x - x_j) + c_j (x - x_j)^2 + d_j (x - x_j)^3, \quad x_j \leq x \leq x_{j+1}.
\]

\subsection*{Natural Cubic Spline Conditions}  
\begin{itemize}
    \item The spline must pass through each given point:
    \[
    S_j(x_j) = f(x_j), \quad S_j(x_{j+1}) = f(x_{j+1}).
    \]
   
    \item The first derivatives must be continuous at each interior point:
    \[
    S_j'(x_{j+1}) = S_{j+1}'(x_{j+1}).
    \]
    \item The second derivatives must also be continuous:
    \[
    S_j''(x_{j+1}) = S_{j+1}''(x_{j+1}).
    \]
    \item The second derivatives at the endpoints are zero:
    \[
    S_0''(x_0) = 0, \quad S_{n-1}''(x_n) = 0.
    \]
\end{itemize}

For simplicity, we set $h_j = x_{j+1} - x_j.$
Since the given points are equally spaced, then \( h_j \) is constant $(h_1=h_2=h_3=h_4 = 1)$.\\

\subsection*{System of Equations for \( c_j \)}  
Using the second derivative continuity condition and the natural spline assumption, we get a tridiagonal system:
\[
2(h_{j-1} + h_j)c_j + h_j c_{j+1} + h_{j-1} c_{j-1} = \frac{3}{h_j}(a_{j+1} - a_j) - \frac{3}{h_{j-1}}(a_j - a_{j-1}).
\]

Since $h_j = h_{j-1} = 1$, this simplifies to:
\[
3(a_{j+1} - a_j) - 3(a_j - a_{j-1}).
\]

%\newpage
For the interior points where the second derivative is continuous $(j=2,3,4)$:
For $j=2$
\[
c_1+4c_2+c_3 = 3(2-2(1)+3)=9.
\]
For $j=3$
\[
c_2+4c_3+c_4=3(3-2(2)+1)=3(0)=0.
\]
For $j=4$
\[
c_3+4c_4+c_5=3(2-2(3)+2)=3(-2)=-6.
\]
Since this is a natural cubic spline, $c_1=0,c_5=0.$\\
This simplifies the system of equations to:
\begin{align*}
4c_2 + c_3 &= 9, \\
c_2 + 4c_3 + c_4 &= 0, \\
c_3 + 4c_4 &= -6.
\end{align*}

\textbf{Now solving:}\\

For \( c_4 \):
\[
c_4 = -\frac{6 - c_3}{4}.
\]

Substituting into the second equation:

\[
c_2 + 4c_3 + \frac{-6 - c_3}{4} = 0.
\]

Multiplying by 4:

\[
4c_2 + 16c_3 - 6 - c_3 = 0.
\]

\[
4c_2 + 15c_3 = 6.
\]

For \( c_3 \):

\[
c_2 = \frac{6 - 15c_3}{4}.
\]

Substituting into the first equation:

\[
4\left(\frac{6 - 15c_3}{4}\right) + c_3 = 9.
\]

\[
6 - 15c_3 + c_3 = 9.
\]

\[
-14c_3 = 3.
\]

\[
c_3 = -\frac{3}{14}.
\]

Solve for \( c_4 \):

\[
c_4 = \frac{-6 - (-3/14)}{4} = \frac{-6 + 3/14}{4} = \frac{-84 + 3}{56} = \frac{-81}{56}.
\]

Solve for \( c_2 \):

\[
c_2 = \frac{6 - 15(-3/14)}{4} = \frac{6 + 45/14}{4} = \frac{84 + 45}{56} = \frac{129}{56}.
\]

\textbf{This yields the final values:}

\[
c_1 = 0, \quad c_2 = \frac{129}{56}, \quad c_3 = -\frac{3}{14}, \quad c_4 = -\frac{81}{56}, \quad c_5 = 0.
\]
\end{document}
